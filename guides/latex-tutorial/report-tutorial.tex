\documentclass[a4paper, onecolumn, 10pt]{report}

% document identification
\title{Tutorial para uso de \LaTeX}
\author{Rui Lopes}
\date{novembro de 2016}

% section for package imports
\usepackage[margin=3.5cm]{geometry}
\usepackage[portuguese]{babel}
\usepackage[utf8]{inputenc}
\usepackage[T1]{fontenc}
\usepackage{hyperref}
\usepackage{listings}
\usepackage{color}
\usepackage{times}
\usepackage{acro}

% acronyms declarations
\DeclareAcronym{WYSIWYG}{
    short = WYSIWYG,
    long = What You See Is What You Get,
    sort = W,
    class = abbrev
}
\DeclareAcronym{WYSIWYM}{
    short = WYSIWYM,
    long = What You See Is What You Mean,
    sort = W,
    class = abbrev
}
\DeclareAcronym{ASCII}{
    short = ASCII,
    long = American Standard Code for Information Interchange,
    sort = I,
    class = abbrev
}
\DeclareAcronym{UTF}{
    short = UTF,
    long = Unicode Transformation Format,
    sort = U,
    class = abbrev
}
\DeclareAcronym{INEM}{
    short = INEM,
    long = Instituto Nacional de Emergência Médica,
    sort = I,
    class = abbrev
}

% color definitions
\definecolor{gray97}{gray}{.97}
\definecolor{gray75}{gray}{.75}
\definecolor{gray45}{gray}{.45}

% style definitions for listings
\lstset{
    frame=Ltb,
    framerule=0pt,
    aboveskip=0.5cm,
    framextopmargin=3pt,
    framexbottommargin=3pt,
    framexleftmargin=0.4cm,
    framesep=0pt,
    rulesep=.4pt,
    backgroundcolor=\color{gray97},
    rulesepcolor=\color{black},
    stringstyle=\ttfamily,
    showstringspaces = false,
    basicstyle=\small\ttfamily,
    commentstyle=\color{gray45},
    keywordstyle=\bfseries,
    numbers=left,
    numbersep=15pt,
    numberstyle=\tiny,
    numberfirstline = false,
    breaklines=true,
    inputencoding=utf8,
    extendedchars=true,
    escapeinside={\%*}{*)},
    literate={á}{{\'a}}1 {ã}{{\~a}}1 {é}{{\'e}}1 {ê}{{\^e}}1 {ó}{{\'o}}1 {ç}{{\c{c}}}1 {Ç}{{\c{C}}}1,
}
\lstnewenvironment{listing}[1][]{\lstset{#1}\pagebreak[0]}{\pagebreak[0]}
\lstdefinestyle{console}{
    basicstyle=\scriptsize\bf\ttfamily,
    backgroundcolor=\color{gray75},
}

\begin{document}

% front cover
\maketitle

\section{Introdução}
De forma a podermos escrever textos onde a real preocupação está no conteúdo,
contrariamente a ferramentas como a Microsoft Word, onde o paradigma é de \ac{WYSIWYG} e também
nos temos de preocupar com os estilos e formatação, a ferramenta de trabalho \LaTeX, derivada
da \TeX, permite a produção de textos científicos e rigorosos através de um paradigma de
\ac{WYSIWYM}. A escolha da criação de relatórios, artigos, \textit{papers}, livros, ... em
\LaTeX, ao invés de outras ferramentas, acontece principalmente pela flexibilidade de escrita que é
fornecida aos autores, permitindo que se alterem conteúdos, por exemplo, num início de um documento
quando este se encontra quase terminado, sem alterar definições de formatação ou estilo - estes
parâmetros são próprios da ferramenta \LaTeX, precisando apenas de ser especificados uma só vez.

\chapter{Estrutura de documentos \LaTeX}
Ao longo deste documento iremos retratar detalhes de estrutura de documentos em \LaTeX, mais em
especial, de documentos da classe \textbf{relatório}, isto é, \texttt{report}.\\
Este documento, que também foi feito em \LaTeX, provém de um ficheiro com extensão \texttt{.tex} que contém
um conjunto de comandos e instruções numa linguagem muito particular. Este código, consideremos, presente
no ficheiro \texttt{report-tutorial.tex} pode ser compilado através de um compilador denominado de
\textbf{pdflatex}. Para compilarmos este código podemos executar o seguinte no terminal:

\begin{lstlisting}[language=sh, style=console]
pdflatex report-tutorial.tex
\end{lstlisting}

\section{Classe de documento e especificações}

Ora, quando criamos um documento em \LaTeX a primeira preocupação, tal como a primeira linha de código a
ser implementada, deverá ser a especificação da \textbf{classe} do documento. Por classe pretende-se
designar se queremos tanto criar um relatório, um artigo, um livro, entre outros, como detalhar quais as
configurações gerais das páginas, como o tamanho de letra base, o número de colunas de texto (entre uma e duas),
tamanho da folha, entre outros... Neste caso em concreto, a primeira linha de código escrita é a que está abaixo.

\lstinputlisting[language=TeX, firstline=1, lastline=1]{report-tutorial.tex}

Por questões de \textbf{boas-práticas} os documentos desta linguagem - embora entre numa norma geral, em termos de
outras linguagens - devem estar organizados da seguinte forma:

\begin{itemize}
    \item Classe de documento e especificações (\texttt{\textbackslash documentclass[options]\{class\}});
    \item Identificação do documento (especificação de título, autores, data, local, ...);
    \item Importação e uso de pacotes (\texttt{\textbackslash usepackage[options]\{package\}});
    \item Caraterização de alguns pacotes (alguns pacotes requerem configurações);
    \item Conteúdo do documento (\texttt{\textbackslash begin\{document\} ... \textbackslash end\{document\}}).
\end{itemize}

\section{Identificação do documento}

Continuando assim na nossa organização temos agora a secção de \textbf{identificação do documento} onde o essencial está
em designar um título, um ou mais autores, uma data, entre outras informações que possam ser acrescentadas. Consideremos apenas
as primeiras três informações, as quais, em \LaTeX, podem ser inseridas através dos comandos \texttt{title}, \texttt{author} e
\texttt{date}, respetivamente. No código do presente documento podemos encontrar a seguinte secção de identificação de documento:

\lstinputlisting[language=TeX, firstline=3, lastline=7]{report-tutorial.tex}

\section{Importação e uso de pacotes}

Para complementar o nosso documento com alguns \textbf{componentes}, como secções de código numa determinada linguagem de
programação, imagens, detalhes linguísticos, entre outros, da mesma forma que em linguagens como a Java, usamos \textbf{pacotes}.
Em termos de \LaTeX, para importar um pacote usamos o comando \texttt{usepackage}, o qual irá questionar as bibliotecas locais do
\LaTeX se os mesmos existem, aplicando-os, se possível.\\
Existem uns tantos pacotes que são importantes considerar na criação de um documento em \LaTeX. Um desses é o \textbf{babel}. Este
pacote permite que o compilador \texttt{pdflatex} tenha noção sobre que regras linguísticas é que este se deve reger aquando da
aplicação, por exemplo, de critérios de \textbf{hifenização}. A hifenização é a truncagem de palavras por sílabas quando estas
excedem a largura da folha onde estas estão escritas. A noção do que é uma sílaba é diferente de idioma para idioma. Através do
pacote \texttt{babel} é possível indicar sobre que regras de ortografia é que se pretende que o compilador se reja. \\
Mais, o pacote \texttt{babel} permite também que se altere o idioma das palavras que o compilador inseriu pelo utilizador, como a
palavra "Capítulo" no caso de um comando \texttt{chapter}, em português, ou de "Conteúdos" no caso do comando
\texttt{tableofcontents} - comando que iremos abordar mais à frente. Mudando o idioma definido nas opções do pacote \texttt{babel}
é possível alterar o idioma de tais palavras, por exemplo, alterando de \texttt{portuguese} para \texttt{italian}, ficando "Capitolo",
ou para \texttt{french}, ficando "Chapitre". \\
Para incluirmos o pacote \texttt{babel} em português podemos assim inserir a seguinte linha de código:

\begin{lstlisting}[language=TeX]
\usepackage[portuguese]{babel}
\end{lstlisting}

O pacote \textbf{inputenc} permite alterar a codificação da entrada do compilador (\textit{input encoding}). Por outras palavras, o \LaTeX,
quando foi criado, não foi desenhado propriamente para idiomas como o português, espanhol, francês, italiano, ... mas antes para o inglês.
O problema disto é que a codificação dos carateres baseia-se em \ac{ASCII}. Acontece que esta codificação não contém carateres como o 'a' com
til (ã) nem o 'o' truncado (ø), por exemplo. Se escrevermos um código em \LaTeX com um carater como 'æ' o mais provável é aparecerem combinações
menos próprias de carateres, ao invés de 'æ'. Para resolver esta situação podemos escrever os carateres através dos comandos \LaTeX, como o
\texttt{\textbackslash acute\{a\}}, que produz um 'á', ou, mais simplesmente, importamos o pacote \texttt{inputenc} e designamos-lhe
uma codificação decente, como a mais usada \textbf{UTF-8} ou \textbf{UTF-16} - \ac{UTF}. No nosso caso, deste documento, estamos a usar a seguinte linha:

\begin{lstlisting}[language=TeX]
\usepackage[utf8]{inputenc}
\end{lstlisting}

Ainda assim, o compilador já pode estar a perceber o intuito do utilizador para um carater fora do mapa do inglês, mas a fonte (tipo de letra)
usada pode não suportar o mesmo. Assim sendo, e para que haja um ajuste da fonte para um determinado conjunto de carateres (como o chinês, por
exemplo), há que identificar uma \textbf{codificação da fonte}, usando o pacote \texttt{fontenc}. Neste documento estamos a usar o tipo 1,
otimizado para os idiomas da Europa Ocidental, como podemos ver de seguida:

\begin{lstlisting}[language=TeX]
\usepackage[T1]{fontenc}
\end{lstlisting}

Outro pacote que poderá ser útil é o de exibição de código de programação. Para isto é possível usar um de dois pacotes bastante conhecidos. A
primeira alternativa é o \texttt{verbatim}. O \texttt{verbatim} permite criar uma área onde o texto em si inserido é escapado (são ignorados
possíveis comandos \LaTeX nele contidos), mas não permite a formatação ou deteção de linguagem de programação para sintaxe com \textit{highlight}.
Pelo contrário, a segunda alternativa, a \texttt{listings} permite a mesma isolação que o \texttt{verbatim} mas com a valência de formatação e destaque
de sintaxe de linguagem de programação. Para usar este pacote apenas temos de escrever a seguinte linha de código:

\begin{lstlisting}[language=TeX]
\usepackage{listings}
\end{lstlisting}

Embora hajam muitos outros pacotes, por último, dos mais importantes, fica o pacote \textbf{acro} que permite fazer uma gestão dos vários acrónimos
ao longo do nosso documento. Assim sendo, fazendo apenas a linha seguinte, podemos usar o pacote \texttt{acronym}:

\begin{lstlisting}[language=TeX]
\usepackage{acro}
\end{lstlisting}

\section{Caraterização de alguns pacotes}
Alguns dos pacotes referidos podem, ou não, exigir que sejam elaboradas algumas configurações para que funcionem. Assim sendo, eis as configurações, caso
necessárias, dos vários pacotes que referimos. \textbf{Nota:} podem haver pacotes que não estão elaborados nestas descrições, no entanto, que sejam usadas
neste documento. Dado que isso poderá acontecer, aconselha-se a analisar o código-fonte (ficheiro \texttt{report-tutorial.tex}) para perceber o que cada
módulo faz.

\subsection{Configuração do pacote \texttt{listings}}

Como foi referido, o pacote \texttt{listings} permite a formatação e destaque da linguagem de programação com a qual o texto em si está escrito. Para tal
é conveniente que se descreva como é que se pretende desenhar as informações de código. Para tal, este pacote permite que se designem as várias informações
através do comando \texttt{lstset}, onde criamos uma possível configuração, como podemos ver no código abaixo:

\begin{lstlisting}[language=TeX]
% style definitions for listings
\lstset{
    frame=Ltb,
    framerule=0pt,
    aboveskip=0.5cm,
    framextopmargin=3pt,
    framexbottommargin=3pt,
    framexleftmargin=0.4cm,
    framesep=0pt,
    rulesep=.4pt,
    backgroundcolor=\color{gray97},
    rulesepcolor=\color{black},
    stringstyle=\ttfamily,
    showstringspaces = false,
    basicstyle=\small\ttfamily,
    commentstyle=\color{gray45},
    keywordstyle=\bfseries,
    numbers=left,
    numbersep=15pt,
    numberstyle=\tiny,
    numberfirstline = false,
    breaklines=true,
}
\end{lstlisting}

No código acima as cores que estão definidas, como se pode ver no ficheiro de código-fonte, estão definidas pela utilização de um outro pacote, o
\texttt{color}, que permite criar a cor que quisermos aplicar a um determinado contexto de utilização. \\
Ao mesmo tempo que neste documento definimos uma configuração-base para as listas, podemos criar configurações alternativas da mesma, criando um estilo
alternativo através do comando \texttt{lstdefinestyle}, seguido do nome que queremos aplicar ao estilo - no nosso caso será \texttt{console}.

\begin{lstlisting}[language=TeX]
\lstdefinestyle{console}{
    basicstyle=\scriptsize\bf\ttfamily,
    backgroundcolor=\color{gray75},
}
\end{lstlisting}

Para usar cada um dos estilos aplicamos a designação \texttt{style=console}, caso queiramos usar o nosso estilo alternativo, no campo de opção do comando
\texttt{lstlisting}. O código-fonte para o código acima é o seguinte, onde no campo de opção estamos a designar que a linguagem que pretendemos aplicar é a
\TeX (mãe da linguagem \LaTeX).

\lstinputlisting[language=TeX, firstline=250, lastline=256]{report-tutorial.tex}

Também podemos ler de um ficheiro presente no mesmo diretório que o ficheiro \texttt{.tex} e até mesmo escolher as linhas, fazendo algo à semelhança do
seguinte:

\begin{lstlisting}[language=TeX]
\lstinputlisting[language=TeX, firstline=250, lastline=256]{report-tutorial.tex}
\end{lstlisting}

\subsection{Configuração do pacote \texttt{acro}}
Para trabalharmos com o pacote \texttt{acro} temos que, à medida que vamos criando novos acrónimos ou abreviaturas, enunciá-las primeiro, antes das suas
utilizações. Para o fazermos, por exemplo, para o acrónimo INEM, temos que criar um bloco de código com o comando \texttt{DeclareAcronym} seguido das
várias informações acerca do mesmo, entre as quais, qual é o seu nome abreviado (\texttt{short}) - INEM -, qual é o seu nome por extenso (\texttt{long}) -
Instituto Nacional de Emergência Médica -, qual é a primeira letra a considerar para termos de ordenação (\texttt{sort}) - letra I - e que tipo de acrónimo
é que estamos a referir (\texttt{class}) - \texttt{abbrev} -, este, podendo ser entre \texttt{abbrev} para abreviatura ou \texttt{nomencl} para nomenclaturas.
Em baixo temos um exemplo de aplicação do acrónimo INEM, como abreviatura:

\begin{lstlisting}[language=TeX]
\DeclareAcronym{INEM}{
    short = INEM,
    long = Instituto Nacional de Emergência Médica,
    sort = I,
    class = abbrev
}
\end{lstlisting}

A partir deste ponto, cada vez que pretendamos usar o acrónimo \ac{INEM} utilizamos o comando \texttt{ac} seguido do nome que atribuímos ao acrónimo. Nesta linha
tal ação já está a ser conduzida, pelo código que vemos em baixo:

\begin{lstlisting}[language=TeX]
(...) pretendamos usar o acrónimo \ac{INEM} utilizamos o comando (...)
\end{lstlisting}

No final dos documentos, ou no início, costumam-se detalhar uma lista de acrónimos usados no mesmo, o qual, em \LaTeX é possível fazer, automaticamente, sem criar
uma página nova, através da seguinte linha de código, onde apenas listamos abreviaturas.

\begin{lstlisting}[language=TeX]
\printacronyms[include-classes=abbrev, name=Acrónimos]
\end{lstlisting}

O resultado desta operação é algo como o seguinte:

\printacronyms[include-classes=abbrev, name=Acrónimos]

Caso queiramos uma lista de nomenclaturas, ao invés do código anterior, devemos escrever antes a linha seguinte:

\begin{lstlisting}[language=TeX]
\printacronyms[include-classes=nomencl, name=Acrónimos]
\end{lstlisting}

\section{Conteúdo do documento} \label{cont-of-doc}
Na criação de um documento, seja de que classe for, há uma determinada hierarquia de regiões do mesmo que convém delinear na sua criação. Consideremos um livro:
quando criamos um livro a primeira preocupação deverá ser criar uma primeira abordagem de abstração para os conteúdos do mesmo, designando-lhe \textbf{partes}.
Cada uma destas partes terá um conjunto de \textbf{capítulos}. Por conseguinte, cada um destes capítulos terá uma \textbf{secção}, depois uma \textbf{subsecção},
entre outros... Em \LaTeX este tipo de hierarquia de regiões é possível ser realizado através dos comandos seguintes:

\begin{itemize}
    \item \texttt{\textbackslash \textbf{part}\{Nome da Parte\}}
    \item \texttt{\textbackslash \textbf{chapter}\{Nome do Capítulo\}}
    \item \texttt{\textbackslash \textbf{section}\{Nome da Secção\}}
    \item \texttt{\textbackslash \textbf{subsection}\{Nome da Subsecção\}}
    \item \texttt{\textbackslash \textbf{subsubsection}\{Nome da Sub-subsecção\}}
    \item \texttt{\textbackslash \textbf{paragraph}\{Parágrafo\}}
    \item \texttt{\textbackslash \textbf{subparagraph}\{Subparágrafo\}}
\end{itemize}

A ordem com que se criam estes itens num ficheiro em \LaTeX permite então, mais tarde, atribuir um número de sequência a cada um destes, detalhando-os
num índice através do comando \texttt{\textbackslash tableofcontents}.

\chapter{Escrita de conteúdos}
Dentro de cada uma das regiões designadas na secção "\nameref{cont-of-doc}" é então possível escrever o mais variado texto - o verdadeiro conteúdo do
documento, razão pela qual este deverá existir. Este conteúdo poderá ter várias formas - poderá ser texto, imagens, tabelas, código, entre outros... Neste
capítulo iremos abordar como interagir em \LaTeX com os tipos de conteúdos mais usuais.

\section{Escrever texto comum}
De todos os tipos de conteúdo possíveis, o \textbf{texto} é, sem dúvida, o mais simples de implementar, dado que apenas é necessário escrever o que se
pretende diretamente no código-fonte \texttt{*.tex}. Acontece que existem alguns pontos essenciais, na implementação do texto, que não são nada lineares. \\

\subsection{Carateres especiais}
Quando queremos executar um novo comando na linguagem \LaTeX enunciamos o mesmo através de uma barra anterior (carater '\textbackslash'). Mas isto causa-nos
um dilema - se nós pretendermos escrever, de facto, o carater '\textbackslash' em texto o compilador irá interpretar que estamos perante um comando, sendo o
nome dele o texto que se segue ao carater. O que fazer, então, para contornar este problema? Para resolver o problema temos de \textbf{escapar} o carater '\textbackslash',
isto é, tirar o significado de \textit{comando} que tem perante o compilador. Assim, ao invés de denotar '\textbackslash' em texto, devemos escrever \texttt{\textbackslash textbackslash}.
Note-se que escrever \textbackslash \textbackslash também não resolve o problema dado que em \LaTeX este comando significa \textit{nova linha}, de forma equivalente
a um \texttt{\textbackslash n} de uma linguagem de programação como a C ou a Java. \\
A mesma coisa acontece para o carater de til - '\textasciitilde'. Em \LaTeX o til tem o significado de um espaço, no entanto, se se pretender escrever em texto, embora
o compilador não se queixe, será necessário escrever \texttt{\textbackslash textasciitilde}. Note-se que o uso de comandos também é essencial quando, não tendo
informações de codificação inseridas no código, se pretendem inserir carateres não-\ac{ASCII}. Por exemplo, para listar as semi-vogais do alfabeto norueguês, não
tendo a codificação para as mesmas (como a \ac{UTF}-8), seria necessário escrever \texttt{\textbackslash r\{a\}} para escrever um \texttt{å}, \texttt{\textbackslash o} para
escrever um \texttt{ø} e \texttt{\textbackslash ae} para escrever um \texttt{æ}. \\
Um outro carater especial e que é necessário escapar são as chavetas. Isto acontece porque nesta linguagem as chavetas funcionam como delimitadores de regiões de contexto
de existência, isto é, limitadores de significado de um determinado comando ou conjunto de comandos. Por exemplo, se escrevermos \texttt{\textbackslash o e} significa, para o compilador,
que o utilizador pretende escrever \o e, sendo que o espaço entre os dois carateres serve como um terminador de \texttt{\textbackslash o} e não como um espaço entre ambos. Uma
forma de resolver esta situação seria aplicando um til ao invés do espaço em código, desta forma \texttt{\textbackslash o~e}, ficando \o~e. No entanto, também podemos resolver
este problema através das chavetas, delimitando o contexto de existência para o comando \texttt{\textbackslash o} numa determinada região, escrevendo simplesmente \texttt{\{\textbackslash o\} e},
ficando assim {\o} e. Para escrever cada uma das chavetas usamos \texttt{\textbackslash \{} ou \texttt{\textbackslash \}}.\\
Por último, o carater '\$' é outro que é importante frisar que não é possível escrevê-lo sem ser no contexto de um comando, dando erro de compilação caso contrário. Isto acontece
porque, de forma semelhante às chavetas, o carater \$ delimita o contexto de existência de \textbf{expressões matemáticas}, assunto que veremos na secção seguinte. Para escrever, assim sendo
o carater \$ devemos usar a barra e escrever o carater pretendido, isto é, escrever \texttt{\textbackslash \$}.

\subsection{Expressões matemáticas}
Como referido no fim da secção anterior, o carater \$ define o início (ou o fim) de uma região onde se descrevem \textbf{expressões matemáticas}. Quando referimos
expressões matemáticas queremos dizer algo como $x = \frac{{ - b \pm \sqrt {b^2 - 4ac} }}{{2a}}$. As expressões matemáticas definem-se então entre \$ singulares, isto é,
\texttt{\$}\textit{expressão}\texttt{\$}. Por exemplo, para fazer a expressão da fórmula resolvente, usou-se o seguinte código:

\begin{lstlisting}[language=TeX]
$x = \frac{{ - b \pm \sqrt {b^2 - 4ac} }}{{2a}}$
\end{lstlisting}

No entanto, fazer \texttt{\$}\textit{expressão}\texttt{\$} não é a única forma de designar uma expressão matemática, sendo que com um \$ singular signfica que prentendemos um
ambiente matemático iniciado de forma \textbf{inline}, isto é, inserido na linha do texto. Acontece que grande parte das vezes as expressões são grandes demais para serem
meramente representadas no tamanho de uma linha de texto comum. Assim sendo, geralmente usam-se \$ duplos para designar um ambiente matemático com o espaço mais adquado para
leitura. Veja-se assim a equação seguinte $$x = \frac{{ - b \pm \sqrt {b^2 - 4ac} }}{{2a}}$$ onde se representa a fórmula resolvente. \\
Ainda assim é pouco rigoroso, num documento científico e de engenharia, representar as expressões matemáticas desta forma, dado que será necessário, em algum momento, referenciar
uma determinada equação num local do texto. Uma forma de resolver este problema é usando o ambiente \texttt{equation}, isto é, usando o comando com o mesmo nome. Assim, quando
escrevemos a expressão matemática, automaticamente é-lhe associado um identificador, que poderá ser usado para referenciação posterior. Note-se, novamente, a equação da fórmula
resolvente em \ref{eq-form-resolvente}.

\begin{equation}
    x = \frac{{ - b \pm \sqrt {b^2 - 4ac} }}{{2a}}  \label{eq-form-resolvente}
\end{equation}

Em termos de código, a fonte para o que está a ser produzido acima está em baixo:

\begin{lstlisting}[language=TeX]
(...) Note-se, novamente, a equação da fórmula resolvente em \ref{eq-form-resolvente}.

\begin{equation}
    x = \frac{{ - b \pm \sqrt {b^2 - 4ac} }}{{2a}}  \label{eq-form-resolvente}
\end{equation}
\end{lstlisting}

\subsection{Listas (de itens e numeradas)}
De forma a poder sintetizar alguns conteúdos, muitas vezes é-nos útil recorrer à criação de \textbf{listas}. As listas, em \LaTeX são muito intuitivas de serem feitas. Existem
três tipos de listas: listas não numeradas com símbolos, listas não numeradas com texto e listas numeradas. As diferenças entre elas estão na utilização dos comandos \texttt{itemize},
\texttt{description} e \texttt{enumerate}, respetivamente. \\
Começando pelo primeiro tipo, o comando \texttt{itemize} permite que seja criada uma lista de marcas, por definição, sendo a marca um ponto. Por exemplo:

\begin{itemize}
    \item Ponto 1
    \item Ponto 2
    \item Ponto 3
\end{itemize}

Cada um dos pontos é indicado através de um comando denominado de \texttt{item}, como podemos ver no código abaixo, descritivo do que está em cima:

\begin{lstlisting}[language=TeX]
\begin{itemize}
    \item Ponto 1
    \item Ponto 2
    \item Ponto 3
\end{itemize}
\end{lstlisting}

Se quisermos mudar as marcas da lista, de pontos para outro tipo de símbolo, basta especificá-lo como a opção dos itens:

\begin{itemize}
    \item[>] Ponto 1
    \item[>] Ponto 2
    \item[>] Ponto 3
\end{itemize}

\begin{lstlisting}[language=TeX]
\begin{itemize}
    \item[>] Ponto 1
    \item[>] Ponto 2
    \item[>] Ponto 3
\end{itemize}
\end{lstlisting}

Quando queremos acrescentar um novo nível de indentação precisamos de instanciar novo bloco de \texttt{itemize}.

\begin{itemize}
    \item[>] Ponto 1
    \item[>] Ponto 2
    \item[>] Ponto 3
    \begin{itemize}
        \item Ponto A
    \end{itemize}
    \item[>] Ponto 4
\end{itemize}

\begin{lstlisting}[language=TeX]
\begin{itemize}
    \item[>] Ponto 1
    \item[>] Ponto 2
    \item[>] Ponto 3
    \begin{itemize}
        \item Ponto A
    \end{itemize}
    \item[>] Ponto 4
\end{itemize}
\end{lstlisting}

Se ao invés de um símbolo pretendermos colocar uma designação (decrição), então usamos o bloco \texttt{description} e no campo opcional indicamos a descrição
pretendida, como em baixo:

\begin{description}
    \item[Programa 1] Ponto 1
    \item[Programa 2] Ponto 2
    \item[Programa 3] Ponto 3
\end{description}

\begin{lstlisting}[language=TeX]
\begin{description}
    \item[Programa 1] Ponto 1
    \item[Programa 2] Ponto 2
    \item[Programa 3] Ponto 3
\end{description}
\end{lstlisting}

Por fim, para fazer uma lista enumerada usamos o bloco de código \texttt{enumerate}, como podemos ver em baixo:\\

\textbf{Há três coisas que me irritam}
\begin{enumerate}
    \item \LaTeX
    \item Listas
    \item Sarcasmo
\end{enumerate}

\begin{lstlisting}[language=TeX]
\textbf{Há três coisas que me irritam}
\begin{enumerate}
    \item \LaTeX
    \item Listas
    \item Sarcasmo
\end{enumerate}
\end{lstlisting}

\subsection{Figuras}

\subsection{Tabelas}

\subsection{Referências textuais}

\chapter{Especificação de Bibliografia - Bib\TeX}

\chapter{Compilação em \LaTeX}

% table of contents
\tableofcontents

\newpage

% call acronym descriptions
\printacronyms[include-classes=abbrev, name=Acrónimos]
\end{document}
